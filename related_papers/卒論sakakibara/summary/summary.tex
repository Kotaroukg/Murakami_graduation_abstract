\chapter{総括}
本研究では,Webブラウザ上でInteractiveに操作可能な,分子動力学法による粒子の振る舞いを視覚化するプログラムを作成した.
このプログラムによる成果を以下に記す.

\begin{enumerate}
  \item 分子動力学法による粒子の動きをシミュレーションし視覚化することによって,クラスタや凝固などの現象を視認することができ直感的な理解が可能となった.
また,粒子をマウスで操作できるため,自分の好きなようにシミュレーションでき理解の向上に繋がると考えられる.
  \item プログラムのJavaScript化を行いWebブラウザ上で動作が可能になり容易に公開,利用することができる.
そのため,分子動力学法を学習する者が手軽に扱うことができ,学習意欲,効率の向上に繋がると考えられる.
  \item 作成したプログラムをライブラリとして保存しプログラムの解説を行うことで,プログラムの継続的な発展を可能にした.
また,今後シミュレーションプログラムを作成する際に今回のプログラムをライブラリとして利用し効率よく作成する事が可能である.
\end{enumerate}



