\begin{abstract}
物理現象を数式だけから理解するのは非常に困難である.
例えば,粒子の運動を解析する分子動力学法(MD)では,ニュートンの運動方程式に従って粒子の運動を数値的に決定していく.
しかし,数値を直接追いかけるだけでは,直感的な理解が得られない.
この際,実際に粒子の運動を視覚化することによって直感的な理解を深め,学習を
促進することができると考えられる.
本研究ではWeb上でInteractiveな操作が可能で,MDの深い理解の助けとなるツールの作成を目的とする.
今回は,Processing言語を用いて,MDにおける粒子の運動を操作・視覚化するプログラムを作成した.
また,Web上で動作させるために,作成したプログラムをJavaScript言語に変換した.
JavaScriptは動的なWebサイト構築に有効な言語の一つである.視覚表示を容易に実現するProcessing 言語においては,JavaScript言語への自動変換が組み込まれおり,Processing言語で作成したプログラムを,ブラウザ上で動作させることができる.
この自動変換には問題点があり,それらの洗い出しと解決策を検討した.
プログラムを作成した結果,クラスタや凝固による粒子の振る舞いを確認することができるようになった.
また,ブラウザ上でInteractiveな操作が可能なため手軽に扱うことができ,今後の学習教材としても有効なものとなった.
そして,ライブラリとしてプログラムの解説を行う事によって今後の継続的な発展に繋がるようにした.
 \end{abstract}
