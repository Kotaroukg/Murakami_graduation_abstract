%スタイル,パッケージの設定
\documentclass[12pt,a4]{jreport}%chapterが使えるスタイル
\usepackage[dvipdfmx]{graphicx}%図の挿入のためのパッケージ
\usepackage{amsmath}
\usepackage{setspace}
\usepackage{amssymb}
\usepackage{ascmac}
\usepackage{framed}

%余白の設定
\setlength{\textheight}{\paperheight}
\setlength{\topmargin}{4.6truemm}
\addtolength{\topmargin}{-\headheight}
\addtolength{\topmargin}{-\headsep}
\addtolength{\textheight}{-60truemm}
\setlength{\textwidth}{\paperwidth}
\setlength{\oddsidemargin}{-0.4truemm}
\setlength{\evensidemargin}{-0.4truemm}
\addtolength{\textwidth}{-50truemm}

%参考文献の設定
\renewcommand{\bibname}{参考文献}


%行間
\setstretch{1.4}

%表紙
\title{卒業論文\\物理現象視覚化ソフトのライブラリ構築}
\author{関西学院大学理工学部\\情報科学科 西谷研究室\\1536 榊原 健}
\date{2015年3月}
\begin{document}
\maketitle
\newpage

%概要
\begin{abstract}
情報や通信に関する技術の総称をICT(Information Communication Technology)という. 近年,ICTを教育に活用する動きが活発であり,このような流れを受けて,政府は2020年を目標に全ての中学,高等学校で,1人1台のタブレット端末を導入したICT授業を実現するという目標を立てている.

タブレット端末の利点として,インタラクティブな教材を使用できる点が挙げられるが, このような教材が効力を発揮する教科の一つに物理がある.物理現象は, 文章や数式による記述では非常に理解しづらいものが存在する.例えば, 高校物理で学習する回折,反射,屈折といった波の性質は上記のような物理現象に相当すると考える.これを直観的に理解させ, 学習を促進する手段として, 物理現象の視覚化が適していると考えた. 
本研究では,反射,屈折,回折といった波の性質の理解を助けるプログラムの作成を目的とした.

回折,反射,屈折の性質を視覚化するためには,多数の波を生成し,それらに重ね合わせの原理を適用しなければならない. そこで複数の点源から波を生成した上で, 重ね合わせの原理によって生じる干渉を視覚化するプログラムを作成した. また,平面波の描写処理が非常に重かったため,描画領域のピクセル数を擬似的に変更することにより,これを大幅に軽減した.

研究の結果,Processing言語を用いて,波の干渉と回折現象を視覚化し,シミュレーションするプログラムを作成した. これにより複数の波が干渉した際の複雑な変位変化や,波長によって回折の度合いが変わる様子を視覚的に理解できるようになった.

今後の課題は,反射角と屈折角の描写が理論通りに描写できなかったため,これを描写できる新たな手法を考えなければならない. また完成したプログラムも,教育現場で使用できる程のデザイン性を有していないため,さらなる改良が必要である.

 \end{abstract}


%目次
\tableofcontents

%本文
\chapter{序論}
情報や通信に関する技術の総称をICT(Information Communication Technology)というが,近年ICTを教育に活用する動きが活発である.
例えば,ICT活用授業による学力向上に関する調査では,ICTを活用した授業をおこなった教員の97.3\%が児童生徒の学力向上に効果があると認めているデータが得られており,このような動きはますます加速すると思われる\cite{simizu}.
このような流れを受けて,政府は2020年を目標に全ての中学,高等学校で,1人1台のタブレット端末を導入したICT授業を実現するという目標を立てている\cite{monbu}. 

タブレット端末を学習に活用することが出来れば, 従来指導が難しいとされていた教科も分かり易く指導を行える可能性が生まれる.そのような状況でタブレット端末の特性を活かした学習コンテンツの作成が喫緊の課題であると考える.

タブレット端末の利点としてインタラクティブな教材を使用できる点が挙げられるが, このような教材が効力を発揮する教科の一つに物理がある.物理現象は, 文章や数式による記述では非常に理解しづらいものが存在する.例えば, 高校物理で学習する干渉,回折,反射,屈折といった波の性質は上記のような物理現象に相当すると考える.これを直観的に理解させ, 学習を促進する手段として, 物理現象の視覚化が適していると考える. 

本研究では,干渉,反射,屈折,回折といった波の性質の理解を助けるプログラムの作成を目的とする.

本書の構成は以下の通りである.まず次章では,どのような波の振る舞いを理解するためのソフトを開発するかを明確にするため, 関連する物理学の知識を簡単に紹介する.次に3章では,開発したソフトの設計や仕様を理解してもらうために,簡単に使用法と振る舞いを示す.4章ではソフトの実装において工夫した点や問題となった課題をどのように解決したかの議論を行っている.5章では本研究で完成させることができなかった反射,屈折の法則の視覚化プログラムの実装状況を示し,処理の問題点について考察している.%序論


\chapter{物理学的基礎知識}
\section{波の要素}
ある点での振動が他の点へと伝わっていく現象を波という.
波を伝える物質を媒質といい,振動によって最初に波が起きた点を波源という.

図\ref{fig:sin}は最も基本的な波である正弦波である.波形の中で最も高い所を山,最も低い所を谷という.隣り合う山同士,谷同士の間の距離を波長λ[$m$]という.波源から山の高さ,もしくは谷の深さを波の振幅という.



\begin{figure}[htbp]
 \begin{center}
  \includegraphics[width=150mm]{../background/lambdasin.png}
 \end{center}
 \caption{波の振幅,波長を示した図.}
 \label{fig:sin}
\end{figure}

\newpage

\section{波の速度と振動数と周期の関係}
1波長分の波が1秒間に発生する回数を振動数$f$[$m$]という.
波の速さ$v$[$m$]は式(\ref{eq:v})と表すことができる.
\begin{eqnarray}
\label{eq:v}
v=fλ
\end{eqnarray}

図\ref{fig:vft}は波の速さ$v$と振動数$f$の関係を示した図である.
\begin{figure}[htbp]
\begin{minipage}[b]{1.0\linewidth}
\centering
\includegraphics[keepaspectratio, scale=0.42]
  {../background/tminute2.png}
 \subcaption{t秒の時の波.}\label{tminute}
 \end{minipage}
 
\begin{minipage}[b]{1.0\linewidth}
\centering
  \includegraphics[keepaspectratio, scale=0.42]
  {../background/t+1minute2.png}
 \subcaption{t+1秒の時の波.}\label{t+1minute}
 \end{minipage}
  
  \caption{波の速さv,振動数fの関係を示した図.}
 \label{fig:vft}
\end{figure}

また1波長分の波が発生するまでに要する時間を周期$T$という.
周期$T$と振動数$f$には式(\ref{eq:t})の関係が成り立つ.
\begin{eqnarray}
\label{eq:t}
T = \frac{1}{f}
\end{eqnarray}


\section{ある地点における正弦波の変位の計算方法}
ある地点における正弦波の変位を計算する際には,以下に挙げる要素が必要である.
\begin{enumerate}
 \item 波源からの距離
 \item 波長
 \item 現在の時間と波源が生成された時間との差
 \end{enumerate}
全ての要素を同時に考慮することは難しいので,1-2の要素と3の要素を分けた上で正弦波の変位の計算方法を説明する.

図\ref{fig:dislam}(a)は,ある地点aの変位y\_aと,波源から地点aまでの距離を表したものである.波源から目標地点(点a)までの距離をd1とする.
図\ref{fig:dislam}(b)は\ref{fig:dislam}(a)の距離d1を波長$λ$で割った余りを距離d2とし,波源から距離d2分離れた場所を地点bとして表したものである.




\begin{figure}[htbp]

\begin{minipage}[b]{1.0\linewidth}
\centering
  \includegraphics[keepaspectratio, scale=0.45]
  {../background/dislam1.png}
 \subcaption{2つの波が重なっている時.}\label{dislam1}
 \end{minipage}
  
  \begin{minipage}[b]{1.0\linewidth}
\centering
  \includegraphics[keepaspectratio, scale=0.45]
  {../background/dislam2.png}
 \subcaption{2つの波が重なった後.}\label{dislam2}
 \end{minipage}
  
  \caption{波源からの距離と波長の値から,ある地点での波の変位を計算する方法.}
 \label{fig:dislam}
\end{figure}











\section{波の独立性と重ね合わせの原理}
教科書P188参考書P104の手法で説明する.


\begin{figure}[htbp]
\begin{minipage}[b]{1.0\linewidth}
\centering
\includegraphics[keepaspectratio, scale=0.45]
  {../background/synwave1.png}
 \subcaption{2つの波が重なる前.}\label{synwave1}
 \end{minipage}
 
\begin{minipage}[b]{1.0\linewidth}
\centering
  \includegraphics[keepaspectratio, scale=0.45]
  {../background/synwave2.png}
 \subcaption{2つの波が重なっている時.}\label{synwave2}
 \end{minipage}
  
  \begin{minipage}[b]{1.0\linewidth}
\centering
  \includegraphics[keepaspectratio, scale=0.45]
  {../background/synwave3.png}
 \subcaption{2つの波が重なった後.}\label{synwave3}
 \end{minipage}
  
  \caption{波の独立性と重ねあわせの原理.}
 \label{fig:synwave}
\end{figure}




\section{反射の法則}
\section{屈折の法則}
\section{回折現象}
\section{ホイヘンスの原理}
上の3つの現象をホイヘンスの原理によって説明する方法を示す.




















\begin{comment}




\chapter{物理的背景}
\section{分子動力学法}
分子動力学法は運動方程式を解く事によって,粒子の振る舞いを解析する手法である\cite{MD}.ニュートンの運動方程式はエネルギー保存則を満たすため,エネルギーが保存される集団において用いられる.
\subsection{Verlet法}
Verlet法は分子動力学法における粒子の座標を逐次的に求める方法であり,式(\ref{eq:verlet})のように表される.


\begin{eqnarray}
\label{eq:verlet}
r(t+h)=2r(t)-r(t-h)+\frac{h^2}{m}f(t)
\end{eqnarray}
ここで,$r(t)$は時刻$t$における粒子の座標,$f(t)$は時刻$t$における粒子に加わっている力,$h$は微小時間,$m$は粒子の質量を表している.
式(\ref{eq:verlet})は,時刻$t$における粒子の座標,時刻$t-h$における粒子の座標,時刻$t$における粒子に加わっている力,粒子の質量の4つの要素から,時刻$t+h$における粒子の座標が求まることを表している.
粒子に加わっている力を常に決定することができれば,Verlet法を継続的に用いることが可能であり,逐次的に粒子の座標を決定し続けることができる.
また,粒子を動かすために速度が必要そうだが,この手法では速度を必要としないという特徴がある.この手法はニュートンの運動方程式から導出されるため,エネルギーが保存される系で有効である.


\subsection{導出}
時刻$t+h$における粒子の座標$r(t+h)$にテイラー展開を行うと式(\ref{eq:verlet2})ができる.

\begin{eqnarray}
\label{eq:verlet2}
r(t+h)=r(t)+h\frac{dr(t)}{dt}+\frac{h^2}{2!}\frac{d^2r(t)}{dt^2}+\frac{h^3}{3!}\frac{d^3r(t)}{dt^3}+...
\end{eqnarray}

$h$は微小時間のため$h^3$以上の項を無視すると
\begin{eqnarray}
\label{eq:verlet5}
r(t+h)=r(t)+h\frac{dr(t)}{dt}+\frac{h^2}{2!}\frac{d^2r(t)}{dt^2}
\end{eqnarray}

$h$を$-h$に置き換えると
\begin{eqnarray}
\label{eq:verlet6}
r(t-h)=r(t)-h\frac{dr(t)}{dt}+\frac{h^2}{2!}\frac{d^2r(t)}{dt^2}
\end{eqnarray}

式(\ref{eq:verlet5})と式(\ref{eq:verlet6})を足し合わせ$r(t+h)$について移項させると式(\ref{eq:verlet3})ができる.

\begin{eqnarray}
\label{eq:verlet3}
r(t+h)=2r(t)-r(t-h)-h^2\frac{d^2r(t)}{dt^2}
\end{eqnarray}

ここでニュートンの運動方程式について考える.$v(t)$を時刻$t$の速度,$r(t)$を時刻$t$の位置とすると式(\ref{eq:newton})ができる.
\begin{eqnarray}
\label{eq:newton}
v(t)=\frac{dr(t)}{dt}
\end{eqnarray}
また,式(\ref{eq:newton})より時刻$t$における加速度$a(t)$は
\begin{eqnarray}
a(t)=\frac{d^2r(t)}{dt^2}
\end{eqnarray}
ここで$f(t)$を時刻$t$に作用する力,$m$を質量とするとニュートンの運動方程式は式(\ref{eq:newton2})となる.
\begin{eqnarray}
\label{eq:newton2}
f(t)=m\frac{d^2r(t)}{dt^2}
\end{eqnarray}
移項させると
\begin{eqnarray}
\label{eq:newton4}
\frac{f(t)}{m}=\frac{d^2r(t)}{dt^2}
\end{eqnarray}

式(\ref{eq:newton4})を式(\ref{eq:verlet3})にを代入すると式(\ref{eq:newton3})となりVerlet法が導出される.
\begin{eqnarray}
\label{eq:newton3}
r(t+h)=2r(t)-r(t-h)+\frac{h^2}{m}f(t)
\end{eqnarray}


\section{Lennard-Jonesポテンシャル}
Lennard-Jonesポテンシャルとは2体間での相互作用ポテンシャルエネルギーを経験則的に表したモデルである\cite{akahon}.$ψ$をポテンシャルエネルギー,$R$を原子間距離,$A$,$B$は任意の定数とすると式(\ref{eq:lennard})のように表される.
\begin{eqnarray}
\label{eq:lennard}
\psi(R)=A\bigg(\frac{1}{R}\bigg)^{12}+B\bigg(\frac{1}{R}\bigg)^6
\end{eqnarray}
式(\ref{eq:lennard})は縦軸をポテンシャルエネルギー,横軸を粒子間距離とすると図\ref{fig:lennard}のような概形になる.


ポテンシャルエネルギーの理解には,安定した状態からずれるとエネルギーが生じるという考え方が適切である.
図\ref{fig:lennard}では極小点が平衡原子間距離であり,双方の粒子が安定した状態であると言える.
原子間距離が安定状態より近くなれば,急激にエネルギーが上がる.これは近づくことで原子同士の影響力が増大するからである.
逆に,安定状態より遠くなると,エネルギーは緩やかに上がっていき,ある高さで上昇が止まる.これは原子同士の影響力が減少するからである.
これに加えて,原子に作用する力について考えていく.ポテンシャルエネルギーを距離で微分することにより作用する力が求まる
.図\ref{fig:lennard}の傾きに注目すると,
平衡原子間距離は極小点であり傾きは0のため力が作用しない.
また,距離が近くなれば傾きは急激に負の値をとり斥力が生まれる.逆に,遠くなれば傾きは正の値をとり引力が生まれるが,傾きが徐々に0に近づくため力が弱まっていく.このように図\ref{fig:lennard}のような概形のポテンシャルエネルギーは,バネのような振る舞いをすることがわかる.
\begin{figure}[htbp]
 \begin{center}
  \includegraphics[width=150mm]{../intro/lennard.png}
 \end{center}
 \caption{Lennard-Jonesポテンシャル.}
 \label{fig:lennard}
\end{figure}

\end{comment}%物理的背景


\chapter{視覚化}
物理現象を数式から理解するのは困難である.視覚化を行うことによって理解の促進に繋がると考えられる.本章では視覚化に用いるプログラミング言語について記述する.
\section{Processing}
Processingはグラフィック機能に特化したオープンソースのプログラミング言語である.文法はJavaによく似ており,デフォルトでスケッチブックが用意されていて容易に図の描画を行う事ができる.
また,ProcessingにはJavaScriptへの自動変換機能が備わっていて,Processingで作成したプログラムをWeb上で動作させる事が可能になる.この自動変換には問題点があり本研究で明らかになったものは\ref{sec:problem}節に記す.

\section{JavaScript}
JavaScriptは動的なWebページを作成するために開発されたプログラミング言語である.Javaと名前が似ているが両者は全く別のものである.
HTML内にJavaScriptのプログラムを埋め込むことにより,Web上でそのプログラムを動かすことが可能である.

\section{変換の問題点}\label{sec:problem}
本研究で明らかになったProcessingからJavaScriptへの変換により生じる問題点を以下に記す.

\subsection{ライブラリの使用}
ライブラリとは便利なパーツのようなものであり,Processingには最初から備わっている公式のライブラリと,インストールすることで利用することができるライブラリがある.
それらはプログラム中で宣言することで容易に利用することができる.本研究のプログラムではスライダー部分にControlP5というスライダーやボタンなどのGUIパーツのライブラリを利用していた.
しかし,ライブラリを利用したプログラムではJavaScriptへの変換後正しく動作しない事が明らかになった.解決策として今回はスライダー部分を自分で実装することにした.スライダーの実装については\ref{sec:slider}節に記す .
\subsection{関数名と変数名が同じ}
関数名と変数名が同じ場合,JavaScriptへの変換後,正しく動作しない.例えば以下のようなプログラムがある.
\begin{screen}
{\small
\begin{verbatim}
void same_name(){
  println(0);
}

void setup(){
 int same_name;
 same_name();
 println(1);  
}
\end{verbatim}}
\end{screen}
このプログラムにはsame\_nameという名前の変数と関数が使われている.これを実行するとProcessingでは01と出力され,JavaScriptでは出力なしという結果になる.

\subsection{JavaScriptの変数型}
Processingは整数のint型変数,浮動小数点のfloat型変数と分けているが,JavaScriptの変数にはそれらの型の概念がなくどちらも同じ変数として扱われる.例えば以下のようなプログラムがある.
\begin{screen}
{\small
\begin{verbatim}
void setup(){
  int integer;
  integer = 3/2;
  println(integer); 
}
\end{verbatim}}
\end{screen}
このプログラムはint型変数intergerに1.5が代入され,intergerの中身を出力する.Processingではint型のため1と出力されるが,JavaScriptでは1.5と出力される.
これはJavaScriptの変数にint型がないことが原因であり,解決策として代入部分を以下のように書けばよい.
\begin{screen}
{\small
\begin{verbatim}
integer = (int)(3/2);
\end{verbatim}}
\end{screen}
このように代入する値をint型にすればJavaScriptでも1と出力され,Processingと揃えることができる.このような値のずれは気付くのが困難なため注意が必要である.

\subsection{translateによる座標の移動}
mousePressedやkeyPressedなどの強制的に呼び出される関数内でtranslateを行うと座標のずれが生じる.例えば以下のようなプログラムがある.

\begin{screen}
{\small
\begin{verbatim}
void setup(){
  size(500,500);
}

void mousePressed(){
  translate(10,10);
  ellipse(130,130,10,10);
}

void draw(){
  background(255);
  fill(0);
  ellipse(100,100,10,10);
}
\end{verbatim}}
\end{screen}
このプログラムは座標(100,100)に常に黒い円があり,クリックするたびに座標(140,140)に黒い円が一瞬描かれるというものである.
JavaScriptへの変換後はクリックするたびに常に描かれている黒い円が座標(10,10)ずつずれていく.解決策として関数mousePressed内を以下のように書き換えればよい.
\begin{screen}
{\small
\begin{verbatim}
void mousePressed(){
  translate(10,10);
  ellipse(130,130,10,10);
  translate(-10,-10);
}
\end{verbatim}}
\end{screen}
このように関数の最後でtranslate(-10,-10)を実行し関数内でtranslateで移動させた座標を戻すことによってJavaScriptの動作をProcessingと一致させることができる.














%視覚化
\chapter{開発結果}

本章では,波の干渉と回折現象をプログラムによって視覚化した実行結果を記述する.

本研究では以下に挙げる物理現象の視覚化プログラムに取り組んだ.
\begin{enumerate}
\item 重ね合わせの原理を適用した波の干渉.
\item 波の回折現象.
\item 反射の法則.
\item 屈折の法則.
\end{enumerate}
これらのうち,1と2の現象は目標通りに視覚化することに成功したが, 3と4の法則の視覚化については未完成の部分を残す形となった. 3と4の未完成部分の考察については\ref{seq:considaration}章に記す.


\section{波の干渉の視覚化}
図\ref{fig:4wave}は指定した複数の点源から生成される円形波の位相変位をシミュレーションし,視覚化を行うプログラムである.
このプログラムにはdrawモードとcheckモードという2つのモードが実装されており,drawモードからcheckモードに移行するにはCキー(checkの頭文字).checkモードからdrawモードに移行するときはDキー(drawの頭文字)を押せばよい.


\begin{figure}[H]
 \begin{center}
  \includegraphics[width=\linewidth]{../result/4wave.png}
 \end{center}
 \caption{波の位相変位を視覚化したプログラムの画面.}
 \label{fig:4wave}
\end{figure}






\subsection{drawモード}
このモードは波源から生じる円形波を描写するモードである. プログラム起動時の画面が図\ref{fig:0wave}である.
画面左側の黒い領域が波を描写する領域,画面右側にはcheckモードで確認するy座標の位置を操作できるスライダーが配置されている. 
図\ref{fig:0wave}の状態で黒い領域上のいずれかの場所をクリックすると,クリックされた座標に波源が生成される.波源が生成されたあと,画面右側にはその波源の周期を変更できるスライダーが生成される.
\begin{figure}[H]
 \begin{center}
  \includegraphics[width=110mm]{../result/0wave.png}
 \end{center}
 \caption{プログラム起動時の画面.}
 \label{fig:0wave}
\end{figure}



図\ref{fig:wave}は1つの波源を生成した後,周期をスライダーによって変更した様子である.
\begin{figure}[H]
 \begin{center}
  \includegraphics[width=110mm]{../result/wave.png}
 \end{center}
 \caption{1つの波の周期をスライダーで変更した画面.}
 \label{fig:wave}
\end{figure}
スライダーが周期を変更できる状態で
Lキー(lambdaの頭文字)を押すと,図\ref{fig:wavechangelambda}のように波の波長を変更できるスライダーに変化する.周期を変更するスライダーに戻したい場合はPキー(periodの頭文字)を押せばよい.

\begin{figure}[H]
 \begin{center}
  \includegraphics[width=110mm]{../result/wavechangelambda.png}
 \end{center}
 \caption{波長を変更できるスライダーに変化させた時の画面.}
 \label{fig:wavechangelambda}
\end{figure}
\newpage
波源は図\ref{fig:5wave}のように5個まで生成できる.
\begin{figure}[htbp]
 \begin{center}
  \includegraphics[width=\linewidth]{../result/5wave.png}
 \end{center}
 \caption{波源を5個生成した時の画面.}
 \label{fig:5wave}
\end{figure}
\subsection{checkモード}
\label{sec:check}
このモードはdrawモードに描写されている赤い線上の変位をリアルタイムに視覚化するモードである.同時刻,同座標で周期,波長が同一な波を生成し,波を生成して360フレーム目の状態をdrawモード,checkモードでそれぞれ描写したのが図\ref{fig:compare}(\subref{drawmode}),(\subref{checkmode})である.


\begin{figure}[H]
\begin{minipage}[b]{1.0\linewidth}
\centering
\includegraphics[keepaspectratio, scale=0.40]
  {../result/drawmode.png}
 \subcaption{drawモード.}\label{drawmode}
 \end{minipage}
 
\begin{minipage}[b]{1.0\linewidth}
\centering
  \includegraphics[keepaspectratio, scale=0.40]
  {../result/checkmode.png}
 \subcaption{checkモード.}\label{checkmode}
 \end{minipage}
  
  \caption{周期5.0,波長40.0の波が生成されてから360フレーム目のdrawモード,checkモード.}
 \label{fig:compare}
\end{figure}
\section{波の回折現象の視覚化}
\label{defraction}
図\ref{fig:diffractiondemo}は障害物の間に一定の間隔で配置された波源から生成される円形波によって,波の回折現象の視覚化を行うプログラムである.

\begin{figure}[H]
 \begin{center}
  \includegraphics[width=120mm]{../result/diffractiondemo.png}
 \end{center}
 \caption{回折現象の視覚化.}
 \label{fig:diffractiondemo}
\end{figure}


プログラムを起動すると図\ref{fig:ds1}の画面になる.
この画面では画面右にあるスライダーで,障害物の隙間の間隔を調節できる.長さの値はスライダー下部に表示されており,図\ref{fig:ds2}のように長さの値に応じて波源の数が決定される.
\begin{figure}[H]
\begin{minipage}{0.5\hsize}
\begin{center}
\includegraphics[width=\linewidth]
  {../result/diffractionstart1.png}
\caption{初期状態}
\label{fig:ds1}
\end{center}
\end{minipage}%
\begin{minipage}{0.5\hsize}
\begin{center}
\includegraphics[width=\linewidth]
  {../result/diffractionstart2.png}
\caption{隙間の間隔を120に設定した様子}
\label{fig:ds2}
\end{center}
\end{minipage}
\end{figure}



これらの画面でSキー(Startの頭文字)を押すと,図\ref{fig:incident}の画面へと切り替わる.図\ref{fig:incident}の画面は画面下部から平行に進行してきた入射波が波源に到達すると,図\ref{fig:diffraction1}のように円形波が生成され,波長に応じた挙動を描写する.
\begin{figure}[H]
\begin{minipage}{0.5\hsize}
\begin{center}
\includegraphics[width=\linewidth]
  {../result/diffraction1.png}
\caption{入射波の描写}
\label{fig:incident}
\end{center}
\end{minipage}%
\begin{minipage}{0.5\hsize}
\begin{center}
\includegraphics[width=\linewidth]
  {../result/diffraction2.png}
\caption{隙間の間隔=120,波長=30.0}
\label{fig:diffraction1}
\end{center}
\end{minipage}
\end{figure}


波長を変えると図\ref{fig:diffraction3}のように,図\ref{fig:diffraction3}と同じ波長の値のまま隙間の間隔を小さくしてシミュレーションすると図\ref{fig:diffraction4}のようになる.
\begin{figure}[H]
\begin{minipage}{0.5\hsize}
\begin{center}
\includegraphics[width=\linewidth]
  {../result/diffraction3.png}
\caption{隙間の間隔=120,波長=200.0}
\label{fig:diffraction3}
\end{center}
\end{minipage}%
\begin{minipage}{0.5\hsize}
\begin{center}
\includegraphics[width=\linewidth]
  {../result/diffraction4.png}
\caption{隙間の間隔=40,波長=200.0}
\label{fig:diffraction4}
\end{center}
\end{minipage}
\end{figure}

\begin{comment}
\section{反射の法則の視覚化}
\label{seq:reflection}
図\ref{fig:reflectionsetumei}は反射面に一定の間隔で配置された波源から生成される円形波の位相情報から反射角を描写するプログラムである.
\begin{figure}[H]
 \begin{center}
  \includegraphics[width=110mm]{../result/reflectionsetumei.png}
 \end{center}
 \caption{プログラムの動作の様子.}
 \label{fig:reflectionsetumei}
\end{figure}

プログラムを起動すると図\ref{fig:reflectionincident}のように入射波となる平行波が画面左上から右下にかけて進んでくる.波源は入射波の速度に応じた間隔で5つ生成される.

入射角が20度の入射波によって各波源が生成され,それらから生じた波を描写したものが図\ref{fig:reflection20}である.図\ref{fig:reflection20}の画面右上の波の描写領域に注目すると3本の線がある.

画面の上端に届いていない黒線は,3個目の波源(入射角と反射角の境界線)から現在の最大変位を結んだものである.各フレームごとに最大変位が変わるため,この線もそれに応じて更新される.

赤線は各フレームごとの最大変位をとる座標の傾きを平均化して描写したものである.\ref{seq:pxcelproblem}節で詳しく取り扱うが,プログラムの仕様上,理論上の最大変位を取得することは出来ないので値を平均化することで精度を高める目的がある.

もう一つの黒線は,反射の法則を満たす角度で3個目の波源から引いた線である.この線と赤線のずれを無くすことを目標としていた.
\begin{figure}[H]
\begin{minipage}{0.5\hsize}
\begin{center}
\includegraphics[width=\linewidth]
  {../result/reflectionincident.png}
\caption{入射波が進行する様子}
\label{fig:reflectionincident}
\end{center}
\end{minipage}%
\begin{minipage}{0.5\linewidth}
\begin{center}
\includegraphics[width=\linewidth]
  {../result/reflectionangle20.png}
\caption{入射角を$20^{\circ}$に設定した波が進行する様子.}
\label{fig:reflection20}
\end{center}
\end{minipage}
\end{figure}

入射波を$20^{\circ}$に設定した結果が図\ref{fig:reflection20finish}, $40^{\circ}$に設定した結果が図\ref{fig:reflection40finish}である.
\begin{figure}[H]
\begin{minipage}{0.5\hsize}
\begin{center}
\includegraphics[width=\linewidth]
  {../result/finishreflectionangle20.png}
\caption{入射角が$20^{\circ}$の時の反射角の結果.}
\label{fig:reflection20finish}
\end{center}
\end{minipage}%
\begin{minipage}{0.5\linewidth}
\begin{center}
\includegraphics[width=\linewidth]
  {../result/finishreflectionangle40.png}
\caption{入射角が$40^{\circ}$の時の反射角の結果.}
\label{fig:reflection40finish}
\end{center}
\end{minipage}
\end{figure}

図\ref{fig:reflection20finish},図\ref{fig:reflection40finish}を見ると僅かながらずれが生じており,本研究ではこのずれを修正することが出来なかった.ずれが生じる原因,改善案については\ref{considaration}章で議論する.

\section{屈折の法則の視覚化}
図\ref{fig:refraction}は屈折率が1.5の媒質に,入射角$30^{\circ}$の入射波が進行した際の屈折角の角度を示すプログラムである.


\begin{figure}[H]
 \begin{center}
  \includegraphics[width=130mm]{../result/refractionangle3015.png}
 \end{center}
 \caption{入射角に対する反射角の角度.}
 \label{fig:refraction}
\end{figure}

この後,媒質に応じた速度の素元波を発生させ,\ref{seq:reflection}節で用いた方法と同じ要領で屈折角を描写しようと考えていたが,反射角の描写が理論通りにできていないので,処理の実装に取り掛かることができなかった.
\end{comment}%結果

\chapter{プログラムの制作過程と解説}
この章では今回作成したプログラムのうち,波の干渉と回折現象を視覚化したプログラムの制作過程の記述と処理の解説を行う.
\section{Processing言語}
本研究のプログラムに使用した言語はProcessing言語である.
Processing言語は以下に挙げるような特徴を有している\cite{ishikawa}.
\begin{enumerate}
\item 基本文法はJavaをベースに記法を簡単化したものであり,
インタラクティブソフトウェアやビジュアルプレゼンテーションを容易に実現することに特化している.
%\item 環境構築が非常に容易かつ無償で行える.これはgithubプラットフォームでの共同開発を試みていた榊さんの利点
\item Windows, iOS, Androidのタブレット端末で用いられる3つのプラットフォーム全てで動作する. %//なんで動作するかを調べる
\end{enumerate}
1の特徴からは,視覚化プログラムの作成に適していると言える.  2の特徴からは,学習者があらゆるタブレット端末を使用してもプログラムが動作することが言える.これらの理由から,本研究で使用するプログラミング言語に最も適していると考えた.

\section{波の干渉の描写}
回折,反射,屈折の性質を可視化するためには多数の波を生成し,それらに重ね合わせの原理を適用しなければならない.そこで複数の点源から波を生成した上で,重ね合わせの原理によって生じる干渉を視覚化するプログラムを作成した.

\subsection{波源から波の代わりとなる円を描写するプログラム}
最初は波の干渉の描写を実現するために,図\ref{fig:missdraw}のように,画面左上から右下へ進行する斜めの線を入射波に見立て,入射波が波源として設定した座標を通過すると,波源から波の代わりとなる円を描写するプログラムを作成した.この後,円が重なった部分の色を変化させたり複数の円の包絡線を太く描写することを考えた. しかし計算アルゴリズムが非常に複雑になることや,処理速度が追いつかないことからこの方法は断念した.

\begin{figure}[H]
 \begin{center}
  \includegraphics[width=70mm]{../implement/miss_draw.png}
 \end{center}
 \caption{プログラムの動作画面.}
 \label{fig:missdraw}
\end{figure}
そこで各ピクセルが波源からの距離,波源の生成された時間,波源から生成される波の波長と周期を基にして自らの地点の位相を計算する手法を考案した.

\subsection{平面波描写のアルゴリズム} 
\label{sec:algo}
アルゴリズムは以下の手順である.
\begin{enumerate}
 \item 平面波の変位計算を何ピクセルごとに行うかを指定する.これにより描写領域のピクセル数を擬似的に変更する.これ以降,説明のために擬似的に変更したピクセルの単位を\emph{擬似ピクセル}と記す.
  \item 描写領域がクリックされると波源の座標情報が追加される.この際1つの波源ごとに,各擬似ピクセルとの距離を計算し,各擬似ピクセルに対応した配列(point\_distance[i][j])に格納している.
  \item 点源とは異なる座標の点における変位の計算を,全ての点源と全ての擬似ピクセルに対しおこない,これによって得られた変位の値を,それぞれの擬似ピクセル上での変位に変換し,各擬似ピクセルごとに設けた配列(point\_[i][j])に足し合わせていく.
  \item 各擬似ピクセルごとの変位の値を基にして,値に応じた色の点を描写する.
 \item 描画領域がクリックされた瞬間のみ手順2を,そうでなければ手順3-4を毎フレームごとに繰り返して平面波の挙動を継続的に描写し続ける.
 \end{enumerate}
 なお,手順2の段階で配列(point\_distance[i][j])に値を格納しているのは,手順3での異なる座標の点における変位の計算をおこなう際,点源と擬似ピクセルの距離が必要であるため,あらかじめ計算させることで処理を軽くするためである.

\subsection{描画領域のピクセル数を擬似的に変更する方法}
\ref{sec:algo}に描写領域のピクセル数を擬似的に変更するとあるが,これを行う事により平面波描写の処理を大幅に軽減することができる.

まず,図\ref{fig:pxcelone}のように8×8ピクセルの描写領域があるとする.
これに対し\ref{sec:algo}のアルゴリズムを適用すると,フレームごとに8×8の64個の座標それぞれに対し変位計算を行うことになる.
一方,図\ref{fig:pxceltwo}のように8×8の描写領域に対し,変位計算を2ピクセルごとに行うとするならば,フレームごとの位相計算は4×4の16個の座標に対してすればよい.つまり計算回数は図\ref{fig:pxcelone}の時と比べ,4分の1回となる.

今回作成したプログラムは描写領域を400×400ピクセルとしているため,
ピクセル数を変更しなければフレームごとに160000回の変位計算を行うことになるが,
4ピクセルごとに変位計算を行うことで計算回数を10000回まで削減した.

\begin{figure}[H]
 \begin{center}
  \includegraphics[width=90mm]{../implement/1pxcel.png}
 \end{center}
 \caption{8$×$8の描写領域に色を塗る場合.}
 \label{fig:pxcelone}
\end{figure}


\begin{figure}[H]
 \begin{center}
  \includegraphics[width=90mm]{../implement/2pxcel.png}
 \end{center}
 \caption{8$×$8の描写領域を擬似的に4$×$4に変更する場合.}
 \label{fig:pxceltwo}
\end{figure}


\newpage
またプログラムの擬似ピクセル数はグローバル変数のpoint\_regulationの値を変えることで変更可能である.
図\ref{fig:1pxceldraw}は擬似ピクセル数を1(変更なし)にした時の波の描写,
 図\ref{fig:8pxceldraw}は擬似ピクセル数を8にした時の波の描写である.
 
 なお,図\ref{fig:1pxceldraw},図\ref{fig:8pxceldraw}以外で論文に用いているプログラムの動作画面は全て擬似ピクセル数を4で設定したものである.


\begin{figure}[H]
 \begin{center}
  \includegraphics[width=140mm]{../implement/pointregulation1.png}
 \end{center}
 \caption{擬似ピクセル数を1に設定した時の波の描写.}
 \label{fig:1pxceldraw}
\end{figure}

\begin{figure}[H]
 \begin{center}
  \includegraphics[width=140mm]{../implement/pointregulation8.png}
 \end{center}
 \caption{画面の画素数を8に設定した時の波の描写.}
 \label{fig:8pxceldraw}
\end{figure}


\subsection{波の色の表現方法}

Processing言語では色を表現する際,RGBカラーモデルとHSBカラーモデルの2つを用途に応じて使用することができる.
RGBカラーモデルは赤(Red),緑(Green),青(Blue)の3つの色を様々な配分で混ぜ合わせることで色を表現し,
HSBカラーモデルは色相(Hue),彩度(Saturation),明度(Brightness)の組み合わせによって色を表現する.

図\ref{fig:hsb}はHSBモードでS(彩度)=100,B(明度)=100に設定し,H(色相)を変更した際の色の変化である.図の右側の数字はH(色相)の値を示している.
このようにHSBカラーモデルではRGBカラーモデルとは異なり,赤,緑,青の3色を1つの値(色相)を入れ替えるだけで表現できる特徴があるため,プログラムではこのモデルを採用した.
\begin{figure}[H]
 \begin{center}
  \includegraphics[width=40mm]{../implement/hsb.png}
 \end{center}
 \caption{HSBモードで色相の値を上下させた時の色の変化.}
 \label{fig:hsb}
\end{figure}


プログラムでは,擬似ピクセルごとに記録された波の変位に応じて描画する点の色を変更する.
この時,波の変位が高い時は赤色,変位が0に近い場合は緑色,変位が低い場合には青色で描写するように処理を行う.
\begin{framed}
{\small
\begin{verbatim}
point[i][j] += number_wave_point;
stroke( (number_wave_point2- point[i][j])*(230/number_wave_point2),100,100);
\end{verbatim}}
\end{framed}
この処理は各擬似ピクセルごとの(変位)を0$\sim$230までの値に変換するものである.

まず(point[i][j])に格納されている擬似ピクセルの変位に現在の波源の数を足す.
例えば点源が3つある場合,各擬似ピクセルの変位は波源の振幅が1であるため-3$\sim$3までの値をとる.ここに点源の数を足すと(point[i][j])の値は0$\sim$6までの値をとり,全て正の数となる.
次に変位の値が大きいほど赤色に近づけるようにするため,点源の数の2倍(number\_wave\_point2)から(point[i][j])を減算する.
最後に点源の数の2倍で割ることで値を0$\sim$1の範囲にしてから230を掛けることで,変位の変化と色の変化を連動させることができる.

\section{Elementary\_wavesクラス} 
このクラスは波源と,波源から生じる波に関する処理を行っている.
 \begin{framed}
{\small
\begin{verbatim}
class Elementary_waves{
  float center_x; float center_y;
  float lambda;
  float period;
  float created_time;
  float period_max; float period_min;
  float lambda_max; float lambda_min;

  Elementary_waves(float _center_x, float _center_y, 
  float _lambda, float _period, 
    float _period_min, float _period_max, 
    float _lambda_min, float _lambda_max){
    	center_x = _center_x; center_y = _center_y;
    	lambda = _lambda;
    	period = _period;
    	period_max = _period_max; period_min = _period_min;
    	lambda_max = _lambda_max; lambda_min = _lambda_min;
    	created_time = 0; 
    }
    float now_displacement_y(float distance, float lambda, 
    float period, float time_adjustment){
    float time_phase_adjustment;
    time_phase_adjustment = 
    ((millis()-time_adjustment)/1000.0)*(360.0 / period); 
    float distance_lambda_adjustment = 
    (distance%lambda)/lambda*(360.0);
    float y = 
    sin(radians(time_phase_adjustment - 
    distance_lambda_adjustment));
    return y;
  }
}
}
\end{verbatim}}
\end{framed}
 
 
 \subsection{メンバ変数}
 クラスElementary\_wavesは波源が持つ様々なパラメータをメンバ変数としている.メンバ変数を以下に示す.
 \begin{description}
 \item[center\_x, center\_y]\mbox{}\\
 波源のx座標,y座標を記録する.
 \item[lambda]\mbox{}\\
 波源から生じる波の波長を記録する.
  \item[period]\mbox{}\\
  波源から生じる波の波長を記録する.
   \item[period\_max, period\_min]\mbox{}\\
   スライダーで周期を調節する際の最大値,最小値を記録する.
    \item[lambda\_max,lambda\_min]\mbox{}\\
    スライダーで波長を調節する際の最大値,最小値を記録する.
     \item[created\_time]\mbox{}\\
     波源が生成された時間を記録する.
 \end{description}

プログラムではクラス配列としてクラスElementary\_wavesのインスタンスを複数生成し,各波源としている.


\subsection{点源とは異なる座標の点における変位の計算の実装}
\ref{sec:calculatey}の方法で任意の擬似ピクセル上の変位計算を行うためのメンバ関数,now\_displacement\_yを実装した.

引数のdistanceにはある波源と,ある擬似ピクセルとの距離を代入する.
lambdaにはある波源から生成される波の波長,periodにはある波源から生成される波の周期を代入する.
time\_adjustmentにはある波源が生成された際に,プログラムを起動してから経過していた時間を代入する.

Processing言語にはプログラムが起動してからのミリ秒(1/1000秒)の数を返り値とするmillis()という関数が備えられている.計算上ミリ秒ではなく秒数に直した方が都合がいいので,millisの値を1000で割っている.

またsin関数の中の値をradians()関数によって角度の単位を「度」からラジアンに変換している.Processing言語のパラメータの単位はラジアンなのでこのような処理を施している.

\section{指定したy座標におけるx座標の変位の描写}
\ref{sec:check}節で説明した,y軸上の変位を描写するアルゴリズムを以下に示す.
\begin{enumerate}
\item 各擬似ピクセルの変位を計算する.
\item 手順1によって得られた値を擬似ピクセルごとに設けられた配列に減算して格納する.
\item drawモードで赤線が描写されている座標の変位を取得し,振幅となる値を掛けた上で点として描写する.
\item 変位を測る目盛りを描写する.
\item 1-4を繰り返す.
\end{enumerate}
手順2で減算する理由は,Processing言語で用いられる座標系を,数学等で用いられる一般的な座標系に変換して描写するためである.

\section{Sliderクラス}
波の干渉の視覚化プログラム内で,榊の作成したプログラムを元に,波源の数をいくら増やした場合でもスライダーの数がそれに応じて増えるようにクラス化した\cite{sakaki}.コードを以下に記す.
\begin{framed}
{\small
\begin{verbatim}
class Sliders{
  int position_x; int position_y;
  int slider_height; int slider_width;
  int digit;
  int number_tickmarks;
  int num_separator;
  float min; float max;
  float cordinate_y;
  String variable_name;
  boolean draw_flag;
  boolean slider_dragged;

  Sliders(int _position_x, int _position_y, int _slider_height,
   int _slider_width, float _min, float _max,
    int _number_tickmarks,int _digit, String _variable_name){
    position_x = _position_x; position_y = _position_y;
    slider_height = _slider_height; slider_width = _slider_width;
    min = _min; max = _max;
    number_tickmarks = _number_tickmarks;
    digit = _digit;
    variable_name = _variable_name;
    draw_flag = false;
    slider_dragged = false;
  }

  float draw_slider(float variable_value){
    if(draw_flag == true){
      num_separator =
       (int)map(variable_value, min, max, number_tickmarks-1,0); 
      cordinate_y = 
      position_y + slider_height - 
      ((float)slider_height / (number_tickmarks-1))*num_separator; 
      fill(255);
      noStroke();                                                       
      rect(position_x, position_y + slider_height, slider_width+50, 30); 
      stroke(1);
      rect(position_x, position_y, slider_width, slider_height);
      fill(0);
      rect(position_x, cordinate_y, slider_width, 
      position_y + slider_height - cordinate_y);
      if(variable_name == "y = "){text(variable_name +
      nf(variable_value*point_regulation,1,digit),
       position_x, position_y + slider_height + 20);
      }else{
        text(variable_name + nf(variable_value,1,digit),
         position_x-20, position_y + slider_height + 20);
      }
      if(mousePressed==false) slider_dragged=false;
      if(mouseX >= position_x & mouseX <= position_x + slider_width 
      & mouseY<=cordinate_y+10 & mouseY>= cordinate_y-10){
        if(mousePressed){
          slider_dragged= true;
        }
      }

if(slider_dragged){
        num_separator+= 
        (int)((cordinate_y - mouseY)/
        ((float)slider_height /(number_tickmarks-1))); 
        num_separator = 
        constrain(num_separator, 0, number_tickmarks-1); 
        if(digit == 0){
          variable_value = 
          (max-min) - round(map(num_separator, 0, 
          number_tickmarks-1, min, max)); 
        }else{
          variable_value = 
          (max+min) - 
          map(num_separator, 0, number_tickmarks-1, min, max);
        }
        return variable_value;
      }else{
        return variable_value;
      }
    }else{
      return variable_value;
    }
  }
}
\end{verbatim}}
\end{framed}

このプログラムを利用する準備として,擬似ピクセル数を格納する変数point\_regulationをグローバル変数で定義しておく必要がある.

 \subsection{メンバ変数}
 クラスSliders内の変数を以下に示す.
 \begin{description}
 \item[position\_x, potision\_y]\mbox{}\\
 スライダー左上のx座標,y座標を記録する.
 \item[slider\_height,slider\_width]\mbox{}\\
 縦横の幅を記録する.
  \item[digit]\mbox{}\\
  返り値の桁数を記録する.
   \item[min,max]\mbox{}\\
   値の最小値,最大値を記録する.
    \item[number\_tickmarks]\mbox{}\\
    返す値を何段階で調節できるようにするかを記録する.
    例えばmin=0,max=10の時,number\_tickmarksを11に設定すると0,1,2...9.10の11段階で値を調節できるスライダーとなる.
    \item[num\_separator]\mbox{}\\
    調節する変数の値がスライダーの何段階目の値に位置しているかを記録する.
        \item[cordinate\_y]\mbox{}\\
        調節している変数の値に応じたy座標を記録する.
     \item[variable\_name]\mbox{}\\
     スライダー下部に表示される,調節する変数名の文字列を記録する.
          \item[draw\_flag]\mbox{}\\
     スライダーを画面に表示するかどうかの判定で利用する.
          \item[slider\_dragged]\mbox{}\\
     スライダーの掴み判定の判定で利用する.
      \item[variable\_value(draw\_sliderの引数)]\mbox{}\\
      調節したい変数を格納している.
 \end{description}
 
  \subsection{処理の流れ}
\begin{enumerate}

\item 調整する変数の値から,その値が下から何段階目に位置するかを計算する
\item 段階の番号から,スライダー部分のy座標を計算する.
\item スライダーを描画する.
\item スライダーの掴み判定(slider\_drugged)を行う.
\item 掴み判定がtrueの場合,
\begin{enumerate}
\item マウスの移動距離に対応した段階数に値を加える.
\item 段階数から調節する変数の値を変更し,その値を返す.
\end{enumerate}
掴み判定がfalseの場合,
\begin{enumerate}
\item 調節する変数の値を変更せずにそのまま返す.
\end{enumerate}
\end{enumerate}

\section{回折現象の視覚化}
波の干渉プログラムで作成した平面波の描写をベースとして作成した.
波源のx座標の間隔は,障害物の間隔に関わらず10ピクセルとしている.実際の物理現象を考えると波源は無限に存在するので波源同士の間隔は存在しないと言えるが,仮に有限であったとしたら波源同士に間隔の差はないと考えたからである.















%実装
\chapter{総括}
今後の課題もここに書く. 箇条書きで「」をすれば反射,屈折を再現できると考えられる.といった風に書く.


\begin{comment}
本研究では,Webブラウザ上でInteractiveに操作可能な,分子動力学法による粒子の振る舞いを視覚化するプログラムを作成した.
このプログラムによる成果を以下に記す.

\begin{enumerate}
  \item 分子動力学法による粒子の動きをシミュレーションし視覚化することによって,クラスタや凝固などの現象を視認することができ直感的な理解が可能となった.
また,粒子をマウスで操作できるため,自分の好きなようにシミュレーションでき理解の向上に繋がると考えられる.
  \item プログラムのJavaScript化を行いWebブラウザ上で動作が可能になり容易に公開,利用することができる.
そのため,分子動力学法を学習する者が手軽に扱うことができ,学習意欲,効率の向上に繋がると考えられる.
  \item 作成したプログラムをライブラリとして保存しプログラムの解説を行うことで,プログラムの継続的な発展を可能にした.
また,今後シミュレーションプログラムを作成する際に今回のプログラムをライブラリとして利用し効率よく作成する事が可能である.
\end{enumerate}

\end{comment}



%総括
\chapter*{謝辞} 
本研究を行うにあたり,終始多大なるご指導,御鞭撻をいただいた西谷滋人教授に対し,深く御礼申し上げます.また,本研究を進めるにあたり,様々な助力,知識の供給を頂きました西谷研究室の同輩,先輩方に心から感謝の意を表します.本当にありがとうございました.(修正済み)%謝辞
\appendix
\chapter{プログラムのソースコード}

\begin{framed}
{\small
\begin{verbatim}
float[][] ball_p = new float[81][2];
float[][] pre_p = new float[81][2];
float[][] ball_f	 = new float[81][2];
float[] mouse_p =new float[2];
float ball_size = 0.2;
float m=0.2;
float h=0.001;
int n = 9;
float slider_n = n;
boolean slider_dragged = false;
boolean slider2_dragged = false;

int click_ball=0;
boolean solid_mode = false;
boolean force_mode = false;

void setup(){
  size(800, 550);
  set_particle();
}

void set_particle(){
  int i, j, k=0;
  for(i=1; i<10; i++){
    for(j=1; j<10; j++){
      if(k==n) break;
      ball_p[k][0] = i;
      ball_p[k][1] = j;
      pre_p[k][0] = i;
      pre_p[k][1] = j;
      k++;
    }
  }

  for(i=0; i<n; i++){
    ball_f[i][0] = 0; 
    ball_f[i][1] = 0;
  } 
}

float distance(float[] a, float[] b){
  return sqrt( sq(a[0]-b[0]) + sq(a[1]-b[1]) );
}

float[] lennard(float p1[], float p2[]){
  float d, force;
  float[] force_xy = new float[2];
  d = distance(p1, p2);
  force = -30 * pow(d,-13) + 30 * pow(d,-7);
  force_xy[0] = force * (p2[0] - p1[0])/d; 
  force_xy[1] = force * (p2[1] - p1[1])/d;
  return force_xy;  
}

float[] verlet(float current[], float previous[], float force[]){
  float[] newpos = new float[2];
  newpos[0]=2*current[0]-previous[0]+h*h/m*force[0];
  newpos[1]=2*current[1]-previous[1]+h*h/m*force[1];
  return newpos;
} 

void reflect(int i){
  if(ball_p[i][0]<ball_size){
  ball_p[i][0] = ball_size*2-ball_p[i][0];
  pre_p[i][0] = ball_size*2-pre_p[i][0];
  }  
  else if(ball_p[i][0]>10-ball_size){
    ball_p[i][0] = (10-ball_size)*2-ball_p[i][0];
    pre_p[i][0] = (10-ball_size)*2-pre_p[i][0];
  }
    
  if(ball_p[i][1]<ball_size){
    ball_p[i][1] = ball_size*2-ball_p[i][1];
    pre_p[i][1] = ball_size*2-pre_p[i][1];
  }
  else if(ball_p[i][1]>10-ball_size && solid_mode)
    ball_p[i][1]=10-ball_size-0.01;
  else if(ball_p[i][1]>10-ball_size){
    ball_p[i][1] = (10-ball_size)*2-ball_p[i][1];
    pre_p[i][1] = (10-ball_size)*2-pre_p[i][1];
  }
}

void draw_particle(int i){
  float v;
  v = distance(ball_p[i], pre_p[i]);
  colorMode(HSB,360,100,100);
  fill(230+5000*v,100,100);  
  ellipse(ball_p[i][0]*50,ball_p[i][1]*50, ball_size*100,ball_size*100);
}
  
void inter_force(int i, int j){
  float[] tmp= new float[2];
  tmp = lennard(ball_p[i], ball_p[j]);
  ball_f[i][0] += tmp[0];
  ball_f[i][1] += tmp[1];
  ball_f[j][0] += -tmp[0];
  ball_f[j][1] += -tmp[1];
}

void calc_verlet(int i){
  float[] tmp= new float[2];
  if(solid_mode) ball_f[i][1]+= 0.00001/(h*h);
  tmp = ball_p[i];
  ball_p[i] = verlet(ball_p[i], pre_p[i],ball_f[i]);
  pre_p[i] = tmp; 
}

void force_reset(int i){
  ball_f[i][0] = 0; 
  ball_f[i][1] = 0;
}

void draw_force(int i){
  float force_sqrt_x, force_sqrt_y;
  strokeWeight(3);
  if(ball_f[i][0]>0) force_sqrt_x = sqrt(abs(ball_f[i][0]));
  else force_sqrt_x = -sqrt(abs(ball_f[i][0]));
  if(ball_f[i][1]>0) force_sqrt_y = sqrt(abs(ball_f[i][1]));
  else force_sqrt_y = -sqrt(abs(ball_f[i][1]));
  line(ball_p[i][0]*50, ball_p[i][1]*50,
    ball_p[i][0]*50+force_sqrt_x, ball_p[i][1]*50+force_sqrt_y);
  strokeWeight(1);
}

void slider(int position_x, int position_y, int slider_height,
  int slider_width, int min, int max, int number_tickmarks){
  
  int num_separator = (int)map(slider_n, min, max, 0, number_tickmarks-1);
  float cordinate_y = position_y + slider_height -
    ((float)slider_height / (number_tickmarks-1))*num_separator;
  fill(255);
  noStroke();
  rect(position_x, position_y + slider_height, slider_width+50, 30); 
  stroke(1);
  rect(position_x, position_y, slider_width, slider_height);
  fill(0);
  rect(position_x, cordinate_y, slider_width, position_y +
    slider_height - cordinate_y);
  
  text( "N="+nf(slider_n,1,0), position_x, position_y + slider_height + 20);
  if(mousePressed==false) slider_dragged=false;
  if(mouseX >= position_x & mouseX <= position_x + slider_width &
    mouseY<=cordinate_y+10 & mouseY>= cordinate_y-10){
    if(mousePressed){
      slider_dragged= true;
    }
   }
   
   if(slider_dragged){
     num_separator+= (int)((cordinate_y - mouseY)/
       ((float)slider_height / (number_tickmarks-1)));
     num_separator = constrain(num_separator, 0, number_tickmarks-1);
     slider_n=map(num_separator, 0, number_tickmarks-1, min, max);
   }
}

void slider2(int position_x, int position_y, int slider_height,
  int slider_width, float min, float max, int number_tickmarks){
  
  int num_separator =
    (int)map(h, min, max, 0, number_tickmarks-1);
  float cordinate_y = position_y + slider_height -
    ((float)slider_height / (number_tickmarks-1))*num_separator;
  fill(255);
  noStroke();
  rect(position_x, position_y + slider_height, slider_width+50, 30); 
  stroke(1);
  rect(position_x, position_y, slider_width, slider_height);
  fill(0);
  rect(position_x, cordinate_y, slider_width, position_y +
    slider_height - cordinate_y);
  
  text( "h="+nf(h,1,4), position_x, position_y + slider_height + 20);
  
  if(mousePressed==false) slider2_dragged=false;
  if(mouseX >= position_x & mouseX <= position_x + slider_width &
    mouseY<=cordinate_y+10 & mouseY>= cordinate_y-10){
    if(mousePressed){
      slider2_dragged= true;
    }
   }
   
   if(slider2_dragged){
     num_separator+= (int)((cordinate_y - mouseY)/
       ((float)slider_height / (number_tickmarks-1)));
     num_separator = constrain(num_separator, 0, number_tickmarks-1);
     h=map(num_separator, 0, number_tickmarks-1, min, max);
   }
}

void mousePressed() {
  int i;
  for(i=0; i<n; i++){
    if(mouseX-10-20<=ball_p[i][0]*50 & mouseX-10+20>= ball_p[i][0]*50 &
    mouseY-10-20<=ball_p[i][1]*50 & mouseY-10+20>= ball_p[i][1]*50){
      click_ball = i;
      fill(0);
      //Processing
      ellipse(pre_p[i][0]*50,pre_p[i][1]*50,ball_size*100,ball_size*100); 
      
      /*JavaScript
      ellipse(10+pre_p[i][0]*50,10+pre_p[i][1]*50,
        ball_size*100,ball_size*100); 
      */
      mouse_p[0] = mouseX;
      mouse_p[1] = mouseY;
      return;
    }
  }
  click_ball = -1;
}

void mouseReleased(){
  if(click_ball != -1){
    ball_f[click_ball][0] +=( mouseX - mouse_p[0])*0.0001/(h*h);
    ball_f[click_ball][1] += (mouseY - mouse_p[1])*0.0001/(h*h);
  }
}

void keyPressed() {  
  if (key == 's'||key == 'S') {
    if(solid_mode) solid_mode = false;
    else solid_mode = true;
  }
      if (key == 'f'||key == 'F') {
    if(force_mode) force_mode = false;
    else force_mode = true;
  }
  if (key == 'r'||key == 'R') set_particle();
}

void draw(){
  colorMode(RGB,256);
  int i;
  int j;
  translate(10,10);
  if (mousePressed == false){
    if(n!=slider_n) {
      n=(int)slider_n;
      set_particle();
    }
    background(255);
    text("Press R to Reset  ",550,35);
    text("Press S to Solid-Mode   "+solid_mode,550,50);
    text("Press F to Force-Mode  "+force_mode,550,65);
    fill(255,255,255);
    rect(0, 0, 500, 500);
    
    
    for( i=0; i<n; i++){
      reflect(i);
      draw_particle(i);
    }
    
    for( i=0; i<n-1; i++){
      for( j=i+1; j<n; j++){
        inter_force(i, j);
      }
    }
    
    for( i=0; i<n; i++){
      if(force_mode){
        draw_force(i);
      }
      calc_verlet(i); 
    }
    
    for(i=0; i<n; i++){
      force_reset(i);  
    }

  }
  
  translate(-10, -10);
  colorMode(RGB,256);
  slider(600, 100, 300, 20, 1, 81, 81);
  slider2(700, 100, 300, 20, 0.0001, 0.01, 100);
  translate(10, 10);
}
\end{verbatim}}
\end{framed}%付録

%参考文献

\begin{thebibliography}{9}
\bibitem{MD}神山新一著,佐藤明,「分子動力学シミュレーション」(朝倉書店,1997).
\bibitem{akahon}西谷滋人著,「固体物理の基礎」(森北出版,2006).
\end{thebibliography}

\end{document}
%終了