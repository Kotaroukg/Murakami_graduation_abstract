\chapter{序論}

原子レベルの物理現象を数式だけから理解するのは非常に困難である.
例えば,粒子の動きを解析する分子動力学法では,ニュートンの運動方程式に従い粒子の運動を決定する.
それらは,数値の羅列で表され直感的な理解が得られない.
学習において自分の頭の中でイメージを作ることは重要であり,理解を助長することができると考えられる.
そこで,本研究ではグラフィック機能に特化したProcessingを用いて分子動力学法での粒子の振る舞いの視覚化を行い,
Interactiveな操作が可能なプログラムを作成する.
また,それらを手軽に扱えるようにWebブラウザ上で動作可能なJavaScriptへの変換を行う.
JavaScriptへの変換はProcessing上で行う事ができるがそれには問題点があり,それらの洗い出しと解決策を検討する.
さらに,作成したプログラムをライブラリ化し,解説を行うことで今後の継続的な発展に繋がると考えられる.



