%\documentclass[a4j,twocolumn,uplatex]{jsarticle}
\documentclass[a4j,twocolumn,uplatex]{jarticle}

\usepackage[dvipdfmx]{graphicx}
\usepackage{wrapfig}	
\usepackage[dvipdfmx]{graphicx}
\usepackage{url}
\usepackage{here}
\usepackage{subfigure}
\setlength{\textheight}{275mm}
\headheight 5mm
\topmargin -30mm
\textwidth 185mm
\oddsidemargin -15mm
\evensidemargin -15mm
\pagestyle{empty}
\def\Vec#1{\mbox{\boldmath $#1$}}
\usepackage[dvipdfmx]{graphics}
\usepackage{comment}

\usepackage{subfigure}



\begin{document}
\title{4回生になる前に伝えたい,Emacsを学ぶ意義}
\author{4回生 村上 大貴}
\date{}
\maketitle
\section{Emacsとは}
Emacsを一言で説明するなら"テキストエディタの一種"と言えます.しかしEmacs Lispプログラムを書き込んだり,パッケージを導入することによって自身の機能を拡張することが可能であり,テキストエディタのプラグインはもちろん,webブラウザやメールアカウント,さらにはtwitterアカウントや動画編集機能までEmacs上で動作させることが可能です.これらの多機能性から「Emacsはテキストエディタを超えた,一つの環境である」\cite{rubikiti}とも評されています. Emacs開発が開始されたのは1970年代中盤であり,viと並び,未だに現役である古来のアプリケーションの中では最古の部類です\cite{wiki}.
操作は基本的にキーバインド(いわゆるショートカットコマンド)で行うため,CUI環境で用いられる事が多く.特にLinuxユーザやサーバエンジニア等に広く利用されています\cite{liginc}.

\vspace{-4mm}
\section{Emacsの習得を推進する理由}
Emacsの習得を推進する理由は,Emacsでの操作を学ぶことにより,圧倒的にテキストの編集速度が向上するからです.
Emacsの操作はキーバインド(いわゆるショートカットコマンド)を用いて行うことが基本です.
マウス操作,クリック操作を行ってきた人は,最初は非常に不便に感じるでしょう.しかしこのキーバインド操作に慣れれば,キーボード上のホームポジションから手が離れることを大幅に減らせて,
無駄な動きを極力抑えることが出来るので,あらゆる作業速度を向上させることが出来ます.

また古くから存在し,非常に有名なテキストエディタなので,以後に開発された様々なテキストエディタには,大抵Emacsの基本的なキーバインドに対応した設定やプラグインが存在しています.
よってEmacsの基本的なキーバインドを習得しておけば,他のテキストエディタを使用する場合でも
これまでの経験値を活かし,スピーディーな操作が可能となります. 

このことからテキストエディタとしてのEmacsを学習することは,プログラミング教育で最初にC言語を学ぶことに似ていると言えます. C言語はプログラミング言語の中では比較的理解が難しいですが,古くから存在しているため,様々な言語の元になっています.つまりしっかりと学習さえしていれば様々な言語で応用することが可能であり,プログラミングスキルを大きく向上します. 同じようにEmacsも最初はキーバインドでの操作に戸惑う人も多いと思いますが,前述の通り,様々なテキストエディタでの操作に応用可能であることから,これを身につければ大きくテキスト編集スキルを向上させることが出来るでしょう.


\vspace{-4mm}
\section{Emacsの学習}
\begin{wrapfigure}[14]{l}{.3\linewidth}
\begin{center}
\includegraphics[width=25mm,clip]{Emacs_zissen_nyuumonn.png}
\caption{大竹 智也(2012),『Emacs実践入門 ~思考を直感的にコード化し,開発を加速する~』,技術評論社.}
\label{fig:kite1}
\end{center}
\end{wrapfigure}
先程も述べたようにEmacsの操作はキーバインドを用いて行うことが基本です. まずは基本的なコマンドをまとめたpdfが西谷研の内部サイトに掲載されているので,それを参考に少しずつ慣れていってください.最初はマウスなどを用いた方が早くて楽と思うかもしれないですが,いずれマウスで操作することが煩わしくなるでしょう. 

また基礎を習得するまでの敷居が高いとの記述がネット上で見受けられますが,
入門書として適切だと思われる本を図1に掲載しています.
\vspace{6mm}



\section{Emacsを極めた場合}
Emacsは前述の通り「テキストエディタを超えた,一つの環境である」と評されています. この「環境」という言葉が意味することは「Emacs上でプログラマーがこなさなければならない全ての仕事を完結させることが出来る.」と言えます.
Emacs上で全ての仕事を完結出来るので,画面上のウィンドウは1つで済み,ブラウザのフォーム入力,メールの作成,プログラムのコーティングも全て同じコマンドでこなすことが可能となります.
これだけでも時間の短縮に繋がりますが,更に自らEmacs Lispプログラムを書き込んだり,パッケージを導入することによって,自由自在に機能を拡張することが可能であることから,自分にとって"最強のテキストエディタ"なるものを作成可能です. もちろんこれらを素早く使いこなすにはそれ相応の努力が必要となりますが,その暁にはタスクを圧倒的かつ効率的に処理出来るようになると言えるでしょう.


\vspace{-4mm}
\begin{thebibliography}{9}

\bibitem{rubikiti} るびきち (2014)「1 なぜEmacsをお勧めするのか?」\url{http://rubikitch.com/2014/08/28/sd1405/}, (2016,9,29).

\bibitem{wiki} wikipedia 「Emacs」,\url{https://ja.wikipedia.org/wiki/Emacs}, (2016,10,4).
\bibitem{liginc} LIG inc.(2016)「プログラミングにおすすめ!テキストエィタ5選」\url{https://liginc.co.jp/242481}(2016,10,4).






\end{thebibliography}
\end{document}


