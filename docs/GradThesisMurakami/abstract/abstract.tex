\begin{abstract}
情報や通信に関する技術の総称をICT(Information Communication Technology)という. 近年,ICTを教育に活用する動きが活発であり,このような流れを受けて,政府は2020年を目標に全ての中学,高等学校で,1人1台のタブレット端末を導入したICT授業を実現するという目標を立てている.

タブレット端末の利点として,インタラクティブな教材を使用できる点が挙げられるが, このような教材が効力を発揮する教科の一つに物理がある.物理現象は, 文章や数式による記述では非常に理解しづらいものが存在する.例えば, 高校物理で学習する回折,反射,屈折といった波の性質は上記のような物理現象に相当すると考える.これを直観的に理解させ, 学習を促進する手段として, 物理現象の視覚化が適していると考えた. 
本研究では,反射,屈折,回折といった波の性質の理解を助けるプログラムの作成を目的とした.

回折,反射,屈折の性質を視覚化するためには,多数の波を生成し,それらに重ね合わせの原理を適用しなければならない. そこで複数の点源から波を生成した上で, 重ね合わせの原理によって生じる干渉を視覚化するプログラムを作成した. また,平面波の描写処理が非常に重かったため,描画領域のピクセル数を擬似的に変更することにより,これを大幅に軽減した.

研究の結果,Processing言語を用いて,波の干渉と回折現象を視覚化し,シミュレーションするプログラムを作成した. これにより複数の波が干渉した際の複雑な変位変化や,波長によって回折の度合いが変わる様子を視覚的に理解できるようになった.

今後の課題は,反射角と屈折角の描写が理論通りに描写できなかったため,これを描写できる新たな手法を考えなければならない. また完成したプログラムも,教育現場で使用できる程のデザイン性を有していないため,さらなる改良が必要である.

 \end{abstract}
