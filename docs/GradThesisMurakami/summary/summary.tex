\chapter{総括}
本研究では,複数の波源から生成される波が重ね合わせの原理により干渉する様子,回折現象,反射の法則,屈折の法則の視覚化プログラムの作成に取り組み,前者2つを完成させた.
それにより得られた成果を以下に示す.
\begin{enumerate}
  \item 任意の座標,時間に波源を生成し,生じる円形波の干渉の様子を確認できることや,波源ごとに波長,周期を調整できることから,波の挙動を直感的に理解しやすくなった.
  \item 数式等では特に説明し辛い,回折現象を視覚化することができた.
  \item プログラムのJavaScript化を行うことでタブレット上での動作が可能となり,今後教育現場で利用できる可能性を持たせた.
\end{enumerate}

今後の課題を以下に示す.
\begin{enumerate}
\item 反射角,屈折角を正確に表示できるようにするため,本研究のプログラムとは異なる反射波,屈折波の描写方法を考案しなければならない.
\item 教材として利用するには,スライダーの形等,デザイン性に欠ける部分がありこれらの改善を行わなければならない.
\end{enumerate}



