\chapter{序論}
情報や通信に関する技術の総称をICT(Information Communication Technology)というが,近年ICTを教育に活用する動きが活発である.
例えば,ICT活用授業による学力向上に関する調査では,ICTを活用した授業をおこなった教員の97.3\%が児童生徒の学力向上に効果があると認めているデータが得られており,このような動きはますます加速すると思われる\cite{simizu}.
このような流れを受けて,政府は2020年を目標に全ての中学,高等学校で,1人1台のタブレット端末を導入したICT授業を実現するという目標を立てている\cite{monbu}. 

タブレット端末を学習に活用することが出来れば, 従来指導が難しいとされていた教科も分かり易く指導を行える可能性が生まれる.そのような状況でタブレット端末の特性を活かした学習コンテンツの作成が喫緊の課題であると考える.

タブレット端末の利点としてインタラクティブな教材を使用できる点が挙げられるが, このような教材が効力を発揮する教科の一つに物理がある.物理現象は, 文章や数式による記述では非常に理解しづらいものが存在する.例えば, 高校物理で学習する干渉,回折,反射,屈折といった波の性質は上記のような物理現象に相当すると考える.これを直観的に理解させ, 学習を促進する手段として, 物理現象の視覚化が適していると考える. 

本研究では,干渉,反射,屈折,回折といった波の性質の理解を助けるプログラムの作成を目的とする.

本書の構成は以下の通りである.まず次章では,どのような波の振る舞いを理解するためのソフトを開発するかを明確にするため, 関連する物理学の知識を簡単に紹介する.次に3章では,開発したソフトの設計や仕様を理解してもらうために,簡単に使用法と振る舞いを示す.4章ではソフトの実装において工夫した点や問題となった課題をどのように解決したかの議論を行っている.5章では本研究で完成させることができなかった反射,屈折の法則の視覚化プログラムの実装状況を示し,処理の問題点について考察している.