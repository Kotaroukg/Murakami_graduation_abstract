\chapter{序論}
近年,教育にICT(Information Communication Technology)を活用する動きが活発である.(ICTの説明をどこかでしておく)
ICT活用授業による学力向上に関する調査では,ICTを活用した授業を行った教員の97.3\%が授業におけるICT活用が児童生徒の学力向上に効果があることを認めているデータが得られている\cite{simizu}.
ICT活用授業を行うための道具の1つとしてタブレット端末が挙げられるが,政府は2020年を目標に中学および高等学校の全生徒がタブレットを携行するという計画を立てている\cite{monbu}. タブレット端末を学習に活用することが出来れば, 従来指導が難しいとされていた教科も分かり易く指導を行える可能性が生まれる.そのような状況でタブレット端末の特性を活かした学習コンテンツの作成が喫緊の課題であると考える.

タブレット端末の利点としてインタラクティブな教材を使用できる点が挙げられるが, このような教材が効力を発揮する教科の一つに物理がある.物理現象は, 文章や数式による記述では非常に理解しづらいものが存在する.例えば, 高校物理で学習する干渉,回折,反射,屈折といった波の性質は上記のような物理現象に相当すると考える.これを直観的に理解させ, 学習を促進する手段として, 物理 現象の視覚化が適していると考える. 

本研究では,反射,屈折,回折といった波の性質の理解を助けるプログラムの作成を目的とする.
プログラムはインタラクティブソフトウェアやビジュアルプレゼンテーションの作成が容易なProcessing 言語により記述する.