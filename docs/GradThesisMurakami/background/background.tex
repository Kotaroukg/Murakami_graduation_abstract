

\chapter{物理学的基礎知識}
\section{波の要素}
ある点での振動が他の点へと伝わっていく現象を波という.
波を伝える物質を媒質といい,振動によって最初に波が起きた点を波源という.

図\ref{fig:sin}は最も基本的な波である正弦波である.波形の中で最も高い所を山,最も低い所を谷という.隣り合う山同士,谷同士の間の距離を波長λ[$m$]という.波源から山の高さ,もしくは谷の深さを波の振幅という.



\begin{figure}[htbp]
 \begin{center}
  \includegraphics[width=150mm]{../background/lambdasin.png}
 \end{center}
 \caption{波の振幅,波長を示した図.}
 \label{fig:sin}
\end{figure}

\newpage

\section{波の速度と振動数と周期の関係}
1波長分の波が1秒間に発生する回数を振動数$f$[$m$]という.
波の速さ$v$[$m$]は式(\ref{eq:v})と表すことができる.
\begin{eqnarray}
\label{eq:v}
v=fλ
\end{eqnarray}

図\ref{fig:vft}は波の速さ$v$と振動数$f$の関係を示した図である.
\begin{figure}[htbp]
\begin{minipage}[b]{1.0\linewidth}
\centering
\includegraphics[keepaspectratio, scale=0.42]
  {../background/tminute2.png}
 \subcaption{t秒の時の波.}\label{tminute}
 \end{minipage}
 
\begin{minipage}[b]{1.0\linewidth}
\centering
  \includegraphics[keepaspectratio, scale=0.42]
  {../background/t+1minute2.png}
 \subcaption{t+1秒の時の波.}\label{t+1minute}
 \end{minipage}
  
  \caption{波の速さv,振動数fの関係を示した図.}
 \label{fig:vft}
\end{figure}

また1波長分の波が発生するまでに要する時間を周期$T$という.
周期$T$と振動数$f$には式(\ref{eq:t})の関係が成り立つ.
\begin{eqnarray}
\label{eq:t}
T = \frac{1}{f}
\end{eqnarray}


\section{ある地点における正弦波の変位の計算方法}
ある地点における正弦波の変位を計算する際には,以下に挙げる要素が必要である.
\begin{enumerate}
 \item 波源からの距離
 \item 波長
 \item 現在の時間と波源が生成された時間との差
 \end{enumerate}
全ての要素を同時に考慮することは難しいので,1-2の要素と3の要素を分けた上で正弦波の変位の計算方法を説明する.

図\ref{fig:dislam}(a)は,ある地点aの変位y\_aと,波源から地点aまでの距離を表したものである.波源から目標地点(点a)までの距離をd1とする.
図\ref{fig:dislam}(b)は\ref{fig:dislam}(a)の距離d1を波長$λ$で割った余りを距離d2とし,波源から距離d2分離れた場所を地点bとして表したものである.




\begin{figure}[htbp]

\begin{minipage}[b]{1.0\linewidth}
\centering
  \includegraphics[keepaspectratio, scale=0.45]
  {../background/dislam1.png}
 \subcaption{2つの波が重なっている時.}\label{dislam1}
 \end{minipage}
  
  \begin{minipage}[b]{1.0\linewidth}
\centering
  \includegraphics[keepaspectratio, scale=0.45]
  {../background/dislam2.png}
 \subcaption{2つの波が重なった後.}\label{dislam2}
 \end{minipage}
  
  \caption{波源からの距離と波長の値から,ある地点での波の変位を計算する方法.}
 \label{fig:dislam}
\end{figure}











\section{波の独立性と重ね合わせの原理}
教科書P188参考書P104の手法で説明する.


\begin{figure}[htbp]
\begin{minipage}[b]{1.0\linewidth}
\centering
\includegraphics[keepaspectratio, scale=0.45]
  {../background/synwave1.png}
 \subcaption{2つの波が重なる前.}\label{synwave1}
 \end{minipage}
 
\begin{minipage}[b]{1.0\linewidth}
\centering
  \includegraphics[keepaspectratio, scale=0.45]
  {../background/synwave2.png}
 \subcaption{2つの波が重なっている時.}\label{synwave2}
 \end{minipage}
  
  \begin{minipage}[b]{1.0\linewidth}
\centering
  \includegraphics[keepaspectratio, scale=0.45]
  {../background/synwave3.png}
 \subcaption{2つの波が重なった後.}\label{synwave3}
 \end{minipage}
  
  \caption{波の独立性と重ねあわせの原理.}
 \label{fig:synwave}
\end{figure}




\section{反射の法則}
\section{屈折の法則}
\section{回折現象}
\section{ホイヘンスの原理}
上の3つの現象をホイヘンスの原理によって説明する方法を示す.




















\begin{comment}




\chapter{物理的背景}
\section{分子動力学法}
分子動力学法は運動方程式を解く事によって,粒子の振る舞いを解析する手法である\cite{MD}.ニュートンの運動方程式はエネルギー保存則を満たすため,エネルギーが保存される集団において用いられる.
\subsection{Verlet法}
Verlet法は分子動力学法における粒子の座標を逐次的に求める方法であり,式(\ref{eq:verlet})のように表される.


\begin{eqnarray}
\label{eq:verlet}
r(t+h)=2r(t)-r(t-h)+\frac{h^2}{m}f(t)
\end{eqnarray}
ここで,$r(t)$は時刻$t$における粒子の座標,$f(t)$は時刻$t$における粒子に加わっている力,$h$は微小時間,$m$は粒子の質量を表している.
式(\ref{eq:verlet})は,時刻$t$における粒子の座標,時刻$t-h$における粒子の座標,時刻$t$における粒子に加わっている力,粒子の質量の4つの要素から,時刻$t+h$における粒子の座標が求まることを表している.
粒子に加わっている力を常に決定することができれば,Verlet法を継続的に用いることが可能であり,逐次的に粒子の座標を決定し続けることができる.
また,粒子を動かすために速度が必要そうだが,この手法では速度を必要としないという特徴がある.この手法はニュートンの運動方程式から導出されるため,エネルギーが保存される系で有効である.


\subsection{導出}
時刻$t+h$における粒子の座標$r(t+h)$にテイラー展開を行うと式(\ref{eq:verlet2})ができる.

\begin{eqnarray}
\label{eq:verlet2}
r(t+h)=r(t)+h\frac{dr(t)}{dt}+\frac{h^2}{2!}\frac{d^2r(t)}{dt^2}+\frac{h^3}{3!}\frac{d^3r(t)}{dt^3}+...
\end{eqnarray}

$h$は微小時間のため$h^3$以上の項を無視すると
\begin{eqnarray}
\label{eq:verlet5}
r(t+h)=r(t)+h\frac{dr(t)}{dt}+\frac{h^2}{2!}\frac{d^2r(t)}{dt^2}
\end{eqnarray}

$h$を$-h$に置き換えると
\begin{eqnarray}
\label{eq:verlet6}
r(t-h)=r(t)-h\frac{dr(t)}{dt}+\frac{h^2}{2!}\frac{d^2r(t)}{dt^2}
\end{eqnarray}

式(\ref{eq:verlet5})と式(\ref{eq:verlet6})を足し合わせ$r(t+h)$について移項させると式(\ref{eq:verlet3})ができる.

\begin{eqnarray}
\label{eq:verlet3}
r(t+h)=2r(t)-r(t-h)-h^2\frac{d^2r(t)}{dt^2}
\end{eqnarray}

ここでニュートンの運動方程式について考える.$v(t)$を時刻$t$の速度,$r(t)$を時刻$t$の位置とすると式(\ref{eq:newton})ができる.
\begin{eqnarray}
\label{eq:newton}
v(t)=\frac{dr(t)}{dt}
\end{eqnarray}
また,式(\ref{eq:newton})より時刻$t$における加速度$a(t)$は
\begin{eqnarray}
a(t)=\frac{d^2r(t)}{dt^2}
\end{eqnarray}
ここで$f(t)$を時刻$t$に作用する力,$m$を質量とするとニュートンの運動方程式は式(\ref{eq:newton2})となる.
\begin{eqnarray}
\label{eq:newton2}
f(t)=m\frac{d^2r(t)}{dt^2}
\end{eqnarray}
移項させると
\begin{eqnarray}
\label{eq:newton4}
\frac{f(t)}{m}=\frac{d^2r(t)}{dt^2}
\end{eqnarray}

式(\ref{eq:newton4})を式(\ref{eq:verlet3})にを代入すると式(\ref{eq:newton3})となりVerlet法が導出される.
\begin{eqnarray}
\label{eq:newton3}
r(t+h)=2r(t)-r(t-h)+\frac{h^2}{m}f(t)
\end{eqnarray}


\section{Lennard-Jonesポテンシャル}
Lennard-Jonesポテンシャルとは2体間での相互作用ポテンシャルエネルギーを経験則的に表したモデルである\cite{akahon}.$ψ$をポテンシャルエネルギー,$R$を原子間距離,$A$,$B$は任意の定数とすると式(\ref{eq:lennard})のように表される.
\begin{eqnarray}
\label{eq:lennard}
\psi(R)=A\bigg(\frac{1}{R}\bigg)^{12}+B\bigg(\frac{1}{R}\bigg)^6
\end{eqnarray}
式(\ref{eq:lennard})は縦軸をポテンシャルエネルギー,横軸を粒子間距離とすると図\ref{fig:lennard}のような概形になる.


ポテンシャルエネルギーの理解には,安定した状態からずれるとエネルギーが生じるという考え方が適切である.
図\ref{fig:lennard}では極小点が平衡原子間距離であり,双方の粒子が安定した状態であると言える.
原子間距離が安定状態より近くなれば,急激にエネルギーが上がる.これは近づくことで原子同士の影響力が増大するからである.
逆に,安定状態より遠くなると,エネルギーは緩やかに上がっていき,ある高さで上昇が止まる.これは原子同士の影響力が減少するからである.
これに加えて,原子に作用する力について考えていく.ポテンシャルエネルギーを距離で微分することにより作用する力が求まる
.図\ref{fig:lennard}の傾きに注目すると,
平衡原子間距離は極小点であり傾きは0のため力が作用しない.
また,距離が近くなれば傾きは急激に負の値をとり斥力が生まれる.逆に,遠くなれば傾きは正の値をとり引力が生まれるが,傾きが徐々に0に近づくため力が弱まっていく.このように図\ref{fig:lennard}のような概形のポテンシャルエネルギーは,バネのような振る舞いをすることがわかる.
\begin{figure}[htbp]
 \begin{center}
  \includegraphics[width=150mm]{../intro/lennard.png}
 \end{center}
 \caption{Lennard-Jonesポテンシャル.}
 \label{fig:lennard}
\end{figure}

\end{comment}